\documentclass{article}
\usepackage{graphicx} % Required for inserting images
\usepackage{hyperref}
\usepackage{listings}
\usepackage{color}

% Example : https://codeforces.com/problemset/problem/266/A 
% So the title would be 266A - Stones on the Table
\title{25A - IQ test} 

% Author must be your full name
\author{Deon Marshal} 

% Date is when you create this report
\date{24 April 2024}

\begin{document}

\maketitle

% There are 4 Sections, Problem, Objective, Solution, Code

% Problem section contains hyperlink to the problem
\section{Problem}

Problem Description : \href{https://codeforces.com/problemset/problem/25/A}

% Objective section contains what is the problem's objective
\section{Objective}

Find out the order of numbers that are different from the others in an odd or even series
Example : 
in the series 1 2 3 5, the number that is different in the series in the context of odd or even is the number in second place because it is even and the rest are odd

% Solution section contains how you approch the problem and your solution
\section{Solution}

In this case, we can define 0 as even, and 1 as odd. When the series has been changed to 0 and 1, 
the usual if else will be carried out where if there are more 1s, then an even number sequence will be printed, 
whereas if there are more 0s then an odd number sequence will be printed.
% Code section contains your solution code

\newpage
\section{Code}

\lstset{language=C++,
        basicstyle=\ttfamily,
        keywordstyle=\color{blue}\ttfamily,
        stringstyle=\color{red}\ttfamily,
        commentstyle=\color{green}\ttfamily,
        morecomment=[l][\color{magenta}]{\#}
}
\begin{lstlisting}

#include <bits/stdc++.h>
#include <iostream>
#include <vector>
#include <algorithm>
using namespace std;
int main() {
    int n;
    cin >> n;
    int arr[n];

    for (int i = 0; i < n; i++) {
        cin >> arr[i];
        arr[i] = arr[i] % 2;
    }
    int ganjil = count(arr, arr+n, 1);
    if (ganjil > n-ganjil) {
        for (int i = 0; i < n; i++) {
            if (arr[i]==0) {
                cout << i+1;
                break;
            }
        }
    } else {
        for (int i = 0; i < n; i++) {
            if (arr[i]==1) {
                cout << i+1;
                break;
            }
        }

    }
    return 0;
}
\end{lstlisting}

\end{document}
