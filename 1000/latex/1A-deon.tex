\documentclass{article}
\usepackage{graphicx} % Required for inserting images
\usepackage{hyperref}
\usepackage{listings}
\usepackage{color}

% Example : https://codeforces.com/problemset/problem/266/A 
% So the title would be 266A - Stones on the Table
\title{1A - Theatre Square} 

% Author must be your full name
\author{Deon Marshal} 

% Date is when you create this report
\date{24 April 2024}

\begin{document}

\maketitle

% There are 4 Sections, Problem, Objective, Solution, Code

% Problem section contains hyperlink to the problem
\section{Problem}

Problem Description : \href{https://codeforces.com/problemset/problem/1/A}

% Objective section contains what is the problem's objective
\section{Objective}

Find out the minimum number of square flagstones that can completely cover a rectangular theater square, without having to cut the flagstone
Example : 
a 6 x 5 theater square requires a minimum of 9 square flagstones to cover it completely even if there are parts of the flagstone that exceed the edge of the theater square

% Solution section contains how you approch the problem and your solution
\section{Solution}

simply put, we only need to use a basic mathematical implementation where if a square area is to be fitted with a square to cover it, 
the length of the wide side and long side of the theater square will be divided by the length of the wide side and long side of the flagstone, 
but because the flagstone is square the difference in length and The width of the flagstone will be ignored. 
Before multiplying the two results to find the total flagstones needed, 
we need to round up the division result because we want to make sure all areas are covered with the number of flagstones as the minimum value.
% Code section contains your solution code

\newpage
\section{Code}

\lstset{language=C++,
        basicstyle=\ttfamily,
        keywordstyle=\color{blue}\ttfamily,
        stringstyle=\color{red}\ttfamily,
        commentstyle=\color{green}\ttfamily,
        morecomment=[l][\color{magenta}]{\#}
}
\begin{lstlisting}

#include <bits/stdc++.h>
#include <iostream>
#include <vector>
#include <algorithm>
#include <cmath>
using namespace std;

int main() {
long long n,m,a;
cin>>n>>m>>a;
long long ans=ceil((double)m/a)*ceil((double)n/a);
cout<<ans ;
return 0;
}

\end{lstlisting}

\end{document}
